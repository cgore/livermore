\section{Conclusions and Final Results}
After all of our tests we arrived at the set of parameters in Table~\ref{tab:final-parameters} for the time series classifier.
In this table the return is the equivalent percentage per-year (\%pa) return provided by the parameters at that setting, and the B\&H ratio is the performance relative to a simplistic buy-and-hold stategy, with 1.0 being equal, less than 1.0 implying an underperforming result over the same period, and greater than 1.0 implying a superior result over the same time period.
The DJIA returned 18.54\%pa over the period investigated here, and we failed to meet that in any of our tests.
For example, 11.06\%pa implies that with all of the other parameters set to their initial default and the reward method set to $a_2$ is equivalent to a savings account yielding 11.06\%pa returns, but underperforming the DJIA itself if we were to merely buy and hold it for the same period of time.
While these results demonstrate the system's ability to learn a complex situation, they are not at a level acceptable for real-world use on the stock market, underperforming the simplistic buy-and-hold strategy.
Instead this system in its current form will only truly be applicable to less interesting problem spaces.

\begin{table}
\begin{center}
\caption{TSC Final Parameters}
\begin{tabular}{|r|l|rr|}
   \hline
   \textbf{parameter} & \textbf{value} & \textbf{return} & \textbf{B\&H ratio}\\
   \hline
   reward method & $a_2$ & 11.06\%pa & 0.67875 \\
   GA threshold, $\theta_{GA}$ & 25 & $\cdots$ & $\cdots$ \\
   crossover probability, $\chi$ & 0.9 & 11.85\%pa & 0.70797 \\
   mutation probability, $\mu$ & 0.15 & 12.74\%pa & 0.74200 \\
   exploration probability, $P_{explr}$ & 0.3 & 13.23\%pa & 0.76145\\
   \hline
\end{tabular}
\label{tab:final-parameters}
\end{center}
\end{table}

TSC would not be a usable real-world system for the stock market unless it were to result in returns in excess of the buy-and-hold strategy, which it did not.
If it were capable of outperforming buy-and-hold then we could use it for automated and unsupervised trading.
As it is, a more effective real-world approach would be to simply purchase an indexing fund.
TSC is no longer useful to us since our interest is specifically automated stock trading, and our research will continue towards other avenues of automated time series analysis and prediction, probably still in the area of evolutionary computation and possibly employing a novel type of LCS.

There are many real-world applications comprising simpler time series than the stock market, and TSC does have a lot of room left to grow still, so continued research by others would be welcomed and potentially fruitful.
TSC demonstrates that an LCS can natively represent a time series under analysis and learn in such an environment:
that demonstration is the most valuable result of this research, perhaps encouraging more attempts at LCS-based time series analysis methods.
