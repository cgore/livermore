\specialhead{Abstract}

% An abstract answers the following questions:

% 1. What is the problem you are solving?
A time series is a sequence of data measured at successive time intervals.
Time series analysis refers to all of the methods employed to understand such data, either with the purpose of explaining the underlying system producing the data or to try to predict future data points in the time series.
Time series analysis is applicable to many problems since there are so many areas that require a more thorough understanding of a time series or the prediction of future values of the time series.
The most typical historical examples of time series would be the weather and the financial markets but there are many more real-world time series problems.

% 2. What is your approach?
An evolutionary algorithm is a non-deterministic method of searching a solution space, and modelled after biological evolutionary processes.
A learning classifier system (LCS) is a form of evolutionary algorithm that operates on a population of mapping rules.
We introduce the time series classifier \emph{TSC}, a new type of LCS that allows for the modeling and prediction of time series data, derived from Wilson's XCSR, an LCS designed for use with real-valued inputs.
Our method works by modifying the makeup of the rules in the LCS so that they are suitable for use on a time series.
All of the operations (mutation, crossover, etc.) applied to the rules also were changed from their traditional forms.

% 3. What are your results (or expected results in some cases)?
We tested TSC on real-world historical stock data.
The system would always return a profit, but not as much as the stock market itself is capable of returning by the utilization of an indexing fund.
The stock market is a notoriously difficult system to model effectively and therefore any positive results at all are notable, and never losing money in the long-term is impressive in itself, often a difficult task for unskilled human traders.

% 4. What is the significance of the work?
Although this initial system appears incapable of producing monetary returns better than that of the stock market itself and may not be the eventual solution, it does perform well enough to demonstrate that the system is capable of learning in a very complex environment.
The inherent complexity of the market makes the system unusable for automated trading, but this approach should prove to be useful in other less challenging real-world time series problems.
