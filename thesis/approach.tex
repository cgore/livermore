\section{Approach and Design of the Time Series Classifier}

\vspace{-\baselineskip}

\subsection{Fundamental Operations}

Our representation of a time series and our approach to their evolutionary methods requires us to be capable of generating multi-dimensional raster paths, where a raster path is a one-dimensional path through a raster space.
This is so that we can run raster paths through a raster space of data, a discrete sampling of data.
A raster space is one that is representable by
$\mathbb{Z}_a \times \mathbb{Z}_b \times \cdots \times \mathbb{Z}_z$.
In other words, all of the dimensions are along sets of finite integers instead of the real numbers.
A common example is raster imagery: a two-dimensional bitmap of size $m \times n$ can be viewed as a complete representation of the two-dimensional raster space of $\mathbb{Z}_m \times \mathbb{Z}_n$.
A multidimensional matrix can therefore fully represent these spaces, instead of merely being samplings of the real space, although we are using these raster spaces for sampling of real data in our approach.
We form a useful sample of the data for further analysis and classification by TSC by generating paths through the data and it is these raster paths that the TSC actually classifies the situations with, not with the entire data set which is generally very large.
We will now outline the basic operations we use to generate raster lines.

\subsubsection{The \emph{Sort On} Algorithm}
\index{sort on algorithm@\emph{sort on} algorithm}
This algorithm sorts a sequence $s$ according to the ordering of another sequence $t$, and is outlined in Algorithm~\ref{alg:sort-on}.
\begin{algorithm}[H]
\caption{Sort on.}
\label{alg:sort-on}
\begin{algorithmic}[1]
\INPUT A sequence $s$ to be sorted.
\INPUT A sequence $t$ upon which to sort $s$ with.
\INPUT A comparator $c$ to sort with, typically $>$ or $<$.
\REQUIRE $|s| = n \le |t|$.
\STATE Construct a sequence $u$ containing pairs of the form $u_i = (s_i, t_i)$ as elements, $|u| = n$,
  \begin{equation}
  u = (u_0, \ldots, u_{n-1})
    = \left( (s_0, t_0), \ldots, (s_{n-1}, t_{n-1}) \right).
  \end{equation}
\STATE Sort $u$ using the second elements as the key, using any normal sorting algorithm, giving
  \begin{equation}
  u' = \left( (s'_0, t'_0), \ldots, (s'_{n-1}, t'_{n-1}) \right)
  \end{equation}
  where $t'_0 \le \ldots \le t'_{n-1}$ if we are sorting in ascending order (with the $<$ comparator).
\RETURN $s' = \left( s'_0, \ldots, s'_{n-1} \right)$.
\end{algorithmic}
\end{algorithm}

\subsubsection{The \emph{Sort Order} Algorithm}
\index{sort order algorithm@\emph{sort order} algorithm}
This algorithm returns the re-ordered indices of a sorted sequence, and is outlined in Algorithm~\ref{alg:sort-order}.
For example, if $t = \{4,5,3,9\}$
then the sorted ordering of $t$ would be $\{3,1,0,2\}$
since $t_3 \ge t_1 \ge t_0 \ge t_2$.
\begin{algorithm}[H]
\caption{Sort order.}
\label{alg:sort-order}
\begin{algorithmic}[1]
\INPUT A sequence $t$.
\INPUT A comparator $c$, usually $<$ or $>$.
\LETARROW{$n$} $|t|$.
\STATE Generate $\mathbb{Z}_n = (0, \ldots, n-1)$.
\RETURN The result of the \emph{sort on} algorithm from \S\ref{alg:sort-on} on $s = \mathbb{Z}_n$ with $t$ and the comparator $c$.
\end{algorithmic}
\end{algorithm}

%\subsubsection{The \emph{Map Inverse} Algorithm}
%\index{map inverse algorithm@\emph{map inverse} algorithm}
%This algorithm finds the inverse of a mapping, and is outlined in Algorithm~\ref{alg:map-inverse}.
%\begin{algorithm}[H]
%\caption{map inverse}
%\label{alg:map-inverse}
%\begin{algorithmic}[1]
%\INPUT A map $m \colon \mathbb{Z}_n \rightarrow \mathbb{Z}_n$, represented as a sequence.
%\STATE Validate that $m$ is a valid map.
%  Sort $m$ giving $m'$, and then remove any duplicates from $m'$ giving $m''$.
%  Check that $m'' = \mathbb{Z}_n$ where $n = |m|$.
%\RETURN $m^{-1}$, the inverse of $m$,
%  by calling the \emph{sort on} algorithm from \S\ref{alg:sort-on} with
%  $s = \mathbb{Z}_n$ and $t = m$.
%\end{algorithmic}
%\end{algorithm}

\subsubsection{Rasterized Linear Paths Through Arrays}
Given an array $A$ of rank $r$ and dimensions
$d_0 \times \cdots \times d_{r-1}$,
we wish to pull a one-dimensional list or vector $v$ of values from the array, starting at position
$A_{s_0\,\ldots\,s_{r-1}}$
and finishing at position
$A_{f_0\,\ldots\,f_{r-1}}$,
following a linear path through the array.

As an example consider the $4 \times 6$ array:
$$A = \left(\begin{array}{cccccc}
a & b & c & d & e & f \\
g & h & i & j & k & l \\
m & n & o & p & q & r \\
s & t & u & v & w & x
\end{array}\right).$$

\paragraph{A purely horizontal path.}
The linear path from $A_{0\,0}$ to $A_{0\,5}$ would be composed of the values
$$\left< A_{0\,0}, A_{0\,1}, A_{0\,2}, A_{0\,3}, A_{0\,4}, A_{0\,5} \right>$$
and would be
$$\left< a, b, c, d, e, f \right>$$
as illustrated by
$$A = \left(\begin{array}{cccccc}
\apmk a & \apmk b & \apmk c & \apmk d & \apmk e & \apmk f \\
g & h & i & j & k & l \\
m & n & o & p & q & r \\
s & t & u & v & w & x
\end{array}\right).$$

\paragraph{A purely vertical path.}
The linear path from $A_{0\,0}$ to $A_{3\,0}$ would be composed of the values
$$\left< A_{0\,0}, A_{1\,0}, A_{2\,0}, A_{3\,0} \right>$$
and would be
$$\left< a, g, m, s \right>$$
as illustrated by
$$A = \left(\begin{array}{cccccc}
\apmk a & b & c & d & e & f \\
\apmk g & h & i & j & k & l \\
\apmk m & n & o & p & q & r \\
\apmk s & t & u & v & w & x
\end{array}\right).$$

\paragraph{A traditional diagonal path.}
The linear path from $A_{0\,0}$ to $A_{3\,3}$ would be composed of the values
$$\left< A_{0\,0}, A_{1\,1}, A_{2\,2}, A_{3\,3} \right>$$
and would be
$$\left< a, h, o, v \right>$$
as illustrated by
$$A = \left(\begin{array}{cccccc}
\apmk a & b & c & d & e & f \\
g & \apmk h & i & j & k & l \\
m & n & \apmk o & p & q & r \\
s & t & u & \apmk v & w & x
\end{array}\right).$$

\paragraph{Non-equal diagonal paths.}
The confusing part arises when we are dealing with diagonal paths with unequal steps.
Consider the linear path from $A_{0\,0}$ to $A_{3\,5}$.
We end up with a stair-stepping path through the array:
$$\left(A_{0\,0}, A_{1\,1}, A_{2\,1}, A_{3\,2}, A_{4\,2}, A_{5\,3}\right)$$
and would be
$$\left(a, h, i, p, q, x\right)$$
as illustrated by
$$A = \left(\begin{array}{cccccc}
\apmk a & b & c & d & e & f \\
g & \apmk h & \apmk i & j & k & l \\
m & n & o & \apmk p & \apmk q & r \\
s & t & u & v & w & \apmk x
\end{array}\right).$$

\paragraph{The Raster Line Algorithm.}
This is the algorithm used to determine a linear raster path, and is outlined in Algorithm~\ref{alg:raster-line}.
It returns a list of points that follow the linear path from the starting point $p$ to the ending point $q$.
This is derived from the algorithm for raster conversion of a 3D line as described in \cite{KaufmanShimony1986}.
This should work for any dimensionality.
\begin{algorithm}[H]
\caption{Raster line.}
\label{alg:raster-line}
\begin{algorithmic}[1]
\INPUT a starting point $p$ and a final point $q$, both represented as lists.
\REQUIRE $|p| = |q| \land
   p_i \in \mathbb{N} \forall p_i \in p \land
   q_i \in \mathbb{N} \forall q_i \in q$.
\IF[This is a simple degenerate case.]{$p = q$}
  \RETURN $\{p\}$, a list containing only one element, $p$.
\ENDIF
\LETARROW{$n$} $|p| = |q|$ be the dimensionality.
\LETARROW{$\delta$} $\left\{ |p_0 - q_0|, \ldots, |p_{n-1} - q_{n-1}| \right\}$, $|\delta| = n$.
\LET $o$ be the sorted ordering of $\delta$ by $>$ from the \emph{sort order} algorithm in \S\ref{alg:sort-order}.
\LET $p'$ and $q'$ be $p$ and $q$ respectively, sorted according to $o$.
\IF[We want the starting point to have the lower initial dimension.]
   {$p'_0 \le q'_0$}
   \STATE Swap $p'$ with $q'$.
\ENDIF
\LETARROW{$\delta'$} $\left\{ |p'_0 - q'_0|, \ldots, |p'_{n-1} - q'_{n-1}| \right\}$.
\LETARROW{$s$} $\left( \sgn \left( p'_0 - q'_0 \right), \ldots,
   \sgn\left( p'_{n-1} - q'_{n-1} \right) \right)$,
    where $\sgn$ is the signum function.
\LETARROW{$d$} $\left\{ d_1, \ldots, d_{n-1} \right\}, |d| = n-1$,
   the deciders, where $d_i \leftarrow 2 \delta'_i - \delta'_0 \forall d_i \in d$.
\LETARROW{$a$} $\left\{ a_1, \ldots, a_{n-1} \right\}, |a| = n-1$, the if-increments, $a_i \leftarrow 2 \delta'_i \forall a_i \in a$.
\LETARROW{$b$} $\left\{ b_1, \ldots, b_{n-1} \right\}, |b| = n-1$, the else-increments, $b_i \leftarrow 2 \left( \delta'_i - \delta'_0 \right) \forall b_i \in b$.
\LETARROW{$r$} $\{ p' \}$, initializing the result of the algorithm, an ordered list of points.
\LETARROW{$z$} $p'$, initializing the current point.
\WHILE[After this, we have $r = \left\{ p', \ldots, q' \right\}$.]
  {$z_0 < q'_0$}
  \STATE Increment $z_0$ by 1.
  \FORALL{$d_i \in d$}
    \IF{$d_i < 0$}
      \STATE increment $d_i$ by $a_i$.
    \ELSE[In this case we have $d_i \ge 0$.]
      \STATE increment $d_i$ by $b_i$ and $z_i$ by $s_i$.
    \ENDIF
  \ENDFOR
  \STATE Push a duplicate of $z$ to the back of $r$, so that now $r = \left\{ p', \ldots, z \right\}$.
\ENDWHILE
\STATE  Reorder the coordinate of the points in $r$ according to the original coordinate ordering forming $r'$ by applying the inverse of $o$, which is $o$.
\IF{we originally swapped the start and end points}
  \RETURN the reverse of $r'$.
\ELSE
  \RETURN $r'$.
\ENDIF
\end{algorithmic}
\end{algorithm}

\subsubsection{List Slices}
This function returns a slice from a one-dimensional list; that is, a modular subset of the list, and is outlined in Algorithm~\ref{alg:list-slice}.
For example, a 2-slice of the list  $\{1,2,3,4,5,6,7,8,9\}$ would be the list $\{1,3,5,7,9\}$.
\begin{algorithm}[H]
\caption{List slice.}
\label{alg:list-slice}
\begin{algorithmic}[1]
\INPUT A list of elements $l = \{l_0, l_1, \ldots, l_{|l|}\}$.
\INPUT A positive rational slice size $s$.
\STATE Initialize the resulting list $r \leftarrow nil = \{\}$, initially empty.
\STATE Initialize the moving index $i \leftarrow 0$.
\WHILE{$i < |l|$}
  \IF{$i \in \mathbb{Z}$}
    \STATE Append $l_i$ to the end of $r$.
  \ENDIF
  \STATE $i \leftarrow i + s$.
\ENDWHILE
\RETURN $r$.
\end{algorithmic}
\end{algorithm}


\subsection{Data Representation}
This LCS is intended to operate on a multivariate time series.
The data consists of a single temporal dimension, several positional dimensions, and a single dimension of type.
This is represented as a linked list consisting of multidimensional arrays, where each element in the matrices is a structure.
Each array of structures represents a single time step; the position in the list is the position in time.
The fields of the structures are independent data.
Thus, any specific value in the multivariate time series could be uniquely referenced in the form:
\begin{equation}
\left\{ t, x_0, \ldots, x_{n-1}, \phi \right\}
\end{equation}
where $t$ is the temporal position,
$x_0,\ldots,x_{n-1}$ are the dimensional positions (for $n$ dimensions),
and $\phi$ is the field selector.
It must hold that $\forall x_i \in \mathbb{N^*}$.
The temporal position $t$ specifies a time $t_{current}-t$, and it must also hold that
$t \in \mathbb{N}_0$.

This representation can be simplified: the entries can be single elements instead of full structures, and the arrays themselves can even be reduced to single elements, reducing to a traditional one-dimensional time series, all using the same code.
This is what is done in the examples here, and all tests were performed on one-dimensional time series, although each entry was a structure containing multiple related data.
For our example of market analysis, $t$ is the number of days from present time, and the fields are the opening price, closing price, high price, low price, adjusted closing price, and the volume of the trades for that particular stock at that particular time.

\newpage
\subsection{Rule Representation}
The representation of a single rule is a collection of predicates;
each predicate must match the current situation for the rule to match the situation.
A single predicate consists of an initial and a final position,
each of the form
\begin{equation}
\left\{ t, x_0, \ldots, x_{n-1} \right\},
\end{equation}
a field selector $\phi$, an operator $\omega$, and a range pair consisting of a lower and upper bound $[l,u]$.
The field selector $\phi$ is to be a lexical closure taking only one argument, which is the structure at the position
$\left\{t,x_0,\ldots,x_{n-1}\right\}$.
If the structure is not a structure, but rather a single element, the only value that would usually make sense for $\phi$ would be an identity function:
simple transformative functions would be acceptable otherwise.
Any function that operates in a uniform manner, applied to a single entry, would be an acceptable $\phi$.
The operator $\omega$ is also a lexical closure, and is intended for classification purposes; all $\omega$'s must operate over a one-dimensional vector of data.

If we take the data along the straight line segment from the initial point $A$ to the final point $B$, forming a vector $d$, we can then form $d'$ by applying $\phi$ to each element in $d$:
\begin{equation}
d'_i = \phi \left(d_i\right) \forall d_i \in d.
\end{equation}
The predicate is said to match the data if and only if
\begin{equation}
l \le \omega \left( d' \right) \le u.
\end{equation}
When all of the predicates of the rule match, then the rule matches; the rule then recommends a particular classification or action.

\subsection{Mutation}
\label{sec:mutation}
The approach to mutation of the paths is to restrict the mutation of the line segment to the same line, only allowing the end points to move up or down along that line.
In this method, the alteration of the line segment is minor, and therefore there is very little change in the actual information held by the path.
This is exactly the sort of effect we wish in mutation: small changes.
By only allowing for smaller mutations we do not have the information stored in the rule itself destroyed completely, but instead it is just slightly modified.
\[\xymatrix{\circ \ar@{.}[r] & \bullet \ar@{-}[rrr] & \bullet & \bullet & \bullet }\]
The lower and upper values of the range are altered, but limited by a maximum mutation parameter, and also limited to ensure that the current situation maintains its current classification under the classifier rule.

\subsection{Crossover}
\label{sec:crossover}
We use a marginally-modified form of one-point crossover.
Consider viewing the environment condition of a rule as consisting of several predicates, each possessing an initial point $A$, a final point $B$, a lower bound $l$, an upper bound $u$, a field $\phi$ and an operation $\omega$.
We could choose to view this as a list of the form
\begin{equation}
\left\{
   A_0, B_0, l_0, u_0, \phi_0, \omega_0,
   \ldots,
   A_{p-1}, B_{p-1}, l_{p-1}, u_{p-1}, \phi_{p-1}, \omega_{p-1}
\right\}
\end{equation}
where $p$ is the number of predicates contained in the rule.
Apply one-point crossover on two lists of this form, but insure that both lists break the predicates in the same way.

\subsection{Learning Parameters}
\label{sec:parameters}
There are numerous parameters used in XCS, a few added by XCSR, and a few more still added here.
Choosing their values wisely can be very important in some problem domains unfortunately.
This subsection gives brief descriptions of the important parameters and specifies sensible default values for typical problems.
It is important that any results described should also list the parameter settings used.

\subsubsection{From XCS}
These are the parameters that are present in XCS.
As such, they are also present in XCSR and TSC.

\paragraph{General Parameters}
These are parameters related to the general operation of XCS.

\begin{description}
\item [Maximum total numerosity.]
\index{maximum total numerosity}
\index{N@$N$|see{maximum total numerosity}}
This is $N$ in \cite{ButzWilson}.
It specifies the maximum size of the population in micro-classifiers,
that is, the maximum sum of the numerosities of the classifiers.
This should be a positive integer, normally in the hundreds or at most the thousands.
\item [Learning rate.]
\index{learning rate}
\index{beta@$\beta$|see{learning rate}}
This is $\beta$ in \cite{ButzWilson}.
It is used as the learning rate for the predicted payoff,
prediction error estimate, GA fitness, and action set size estimate
for the classifiers.
This should be in the range $[0.1, 0.2]$ for most problems, and always in the range $[0, 1)$.
\item [Possible actions.]
\index{possible actions}
\index{A@$\mathpzc{A}$|see{possible actions}}
This is $\mathpzc{A}$, the set of all of the possible actions that the classifier rules may take for values of $a$.
\end{description}

\paragraph{Recalculating Fitness}
These parameters are used in XCS while recalculating the fitness of the rules in the population.

\begin{description}
\item [Multiplier parameter.]
\index{multiplier parameter}
\index{alpha@$\alpha$|see{multiplier parameter}}
This is $\alpha$ in \cite{ButzWilson}.
This is the multiplier used in recalculating the fitness of the classifiers in the
\emph{update fitness} algorithm from \S\ref{sec:update-fitness}.
It is usually around 0.1.
\item [Equal error threshold.]
\index{equal error threshold}
\index{epsilon 0@$\epsilon_0$|see{equal error threshold}}
This is $\epsilon_0$ in \cite{ButzWilson}.
This is the threshold used in recalculating the fitness of the classifiers in the
\emph{update fitness} algorithm from \S\ref{sec:update-fitness} to decide if the errors are essentially the same.
It is usually around 1\% of the $\rho$, the reward.
\item [Power parameter.]
\index{power parameter}
\index{nu@$\nu$|see{power parameter}}
This is $\nu$ in \cite{ButzWilson}.
This is the exponent used in recalculating the fitness of the classifiers in the
\emph{update fitness} algorithm from \S\ref{sec:update-fitness}.
It is typically set to 5.
\end{description}

\paragraph{Multi-Step Specific}
These are parameters that are only used in multi-step problems.

\begin{description}
\item [Discount factor.]
\index{discount factor}
\index{gamma@$\gamma$|see{discount factor}}
This is $\gamma$ in \cite{ButzWilson}.
It is the discount factor used in multi-step problems when updating the classifier predictions.
It is typically around 0.71.
\end{description}

\paragraph{GA Specific}
These parameters are only used by the GA within XCS.

\begin{description}
\item [GA Threshold.]
\label{sec:ga-threshold}
\index{GA threshold}
\index{theta GA@$\theta_{GA}$|see{GA threshold}}
This is $\theta_{GA}$ in \cite{ButzWilson}.
The GA is run whenever the average number of generations since the last time the GA was run is greater than this threshold.
It is typically in the range $[25, 50]$, and should always be in $\mathbb{N^*}$.
\item [Crossover probability.]
\label{sec:crossover-probability}
\index{crossover probability}
\index{chi@$\chi$|see{crossover probability}}
This is $\chi$ in \cite{ButzWilson}.
It is the probability of applying the crossover operator while executing the GA.
It is typically  in the range $[0.5, 1.0]$.
\item [Mutation probability.]
\label{sec:mutation-probability}
\index{mutation probability}
\index{mu@$\mu$|see{mutation probability}}
This is $\mu$ in \cite{ButzWilson}.
It is the probability of applying the mutation operator while executing the GA.
It is typically in the range $[0.01, 0.05]$.
\item [Deletion threshold.]
\index{deletion threshold}
\index{theta del@$\theta_{del}$|see{deletion threshold}}
This is $\theta_{del}$ in \cite{ButzWilson}.
It is the threshold for classifier deletion.
If a classifier's experience is greater than this parameter then it may be considered for deletion.
It is typically 20.
\item [Fitness fraction threshold.]
\index{fitness fraction threshold}
\index{delta@$\delta$|see{fitness fraction threshold}}
This is $\delta$ in \cite{ButzWilson}.
It is the fraction of the mean fitness of the population below which the fitness of a classifier may be considered in its probability of deletion.
It is typically around 0.1.
\item [Initial fitness.]
\index{initial fitness}
\index{F I@$F_I$|see{initial fitness}}
This is $F_I$ in \cite{ButzWilson}.
It is used as the initial value of the fitness used by the GA for the newly-created classifiers. 
It is typically only slightly more than zero.
\end{description}

\paragraph{Rule Set Specific}
These parameters deal with the rule set as a whole.

\begin{description}
\item [Minimum subsumption experience.]
\index{minimal subsumption experience}
\index{theta sub@$\theta_{sub}$|see{minimal subsumption experience}}
This is $\theta_{sub}$ in \cite{ButzWilson}.
The experience of a classifier must be greater than this threshold for it to subsume another classifier.
It must hold that $\theta_{sub} \in \mathbb{N^*}$, and typically we have $\theta_{sub} \ge 20$.
\item [Covering probability.]
\label{sec:covering-probability}
\index{covering probability}
\index{P \#@$P_\#$|see{covering probability}}
This is $P_\#$ in \cite{ButzWilson}.
It is the probability of using the covering element in a single attribute.
It is typically around 0.33.
\item [Initial prediction.]
\index{initial prediction}
\index{p I@$p_I$|see{initial prediction}}
This is $p_I$ in \cite{ButzWilson}.
It is used as the initial value of the predicted payoff for the newly-created classifiers.
This is typically slightly more than zero.
\item [Initial prediction error.]
\index{initial prediction error}
\index{epsilon I@$\epsilon_I$|see{initial prediction error}}
This is $\epsilon_I$ in \cite{ButzWilson}.
It is used as the initial value of the estimated prediction error for the newly-created classifiers.
It is typically only slightly more than zero.
\item [Exploration probability.]
\label{sec:exploration-probability}
\index{exploration probability}
\index{P explr@$P_{explr}$|see{exploration probability}}
This is $P_{explr}$ in \cite{ButzWilson}.
It specifies the probability of exploration during the action selection phase.
It is typically around 0.5.
\item [Minimal number of actions.]
\index{minimal number of actions}
\index{theta mna@$\theta_{mna}$|see{minimal number of actions}}
This is $\theta_{mna}$ in \cite{ButzWilson}.
This should be in $\mathbb{N}$, and is typically equal to the number of possible actions, so that complete covering will take place.
\item [Maximum number of steps.]
\index{maximum number of steps}
This is the maximum number of steps that a multistep problem can spend in one trial.
This variable is not mentioned in \cite{ButzWilson}, but it is present in Butz's XCS code written in the C programming language.
\item [GA subsumption?]
\index{GA subsumption?}
\index{doGASubsumption|see{GA subsumption?}}
This is \emph{doGASubsumption} in \cite{ButzWilson}.
It is a boolean parameter specifying if the offspring are to be tested for possible logical subsumption by the parents.
It is usually best to set this to $true$.
\item [Action set subsumption?]
\index{action set subsumption?}
\index{doActionSetSubsumption|see{action set subsumption?}}
This is \emph{doActionSetSubsumption} in \cite{ButzWilson}.
It is a boolean parameter specifying if action sets are to be tested for subsuming classifiers.
It is usually best to set this to $true$.
\end{description}

\subsubsection{From XCSR}
These are the learning parameters that are added to an XCS system by XCSR.
Since our system derives from XCSR, we use these as well.
The variables used here are slightly different from those in a traditional XCSR.

\begin{description}
\item [Problem range.]
\index{problem range}
This is a two-element list of the lower and upper values that the input is expected to lie within.
As the input violates this, this range is expanded automatically.
As an example, if it is known for a specific problem that the input should always lie within the real-valued range $[0,1]$, then this should be set to the list $\{0.0,1.0\}$.
\item [Covering maximum.]
\index{covering maximum}
This is how large of a fraction of the range can be added to both the lower and upper bounds combined in the covering.
The current default value we are using is 0.1.
Thus, if we wish to cover $[0.3,0.5]$,  which has a spread of $0.5-0.3=0.2$, the largest allowable spread would be
$(1 + covering_{maximum}) spread = (1+0.1) 0.2 = 0.22$.
\item [Mutation maximum.]
\index{mutation maximum}
This is how large of a fraction of the range may be added or subtracted from the lower and upper bounds in the mutation method.
The current default value we are using is 0.1.
For example, if we are mutating a rule which matches the bounds $[0.3,0.72]$, which has a spread of $0.72-0.3=0.42$, we would have a maximum change of 0.042, so our mutated rule would now match bounds determined randomly from $[0.3 \pm 0.042, 0.72 \pm 0.042]$, but enforced to be within the problem bounds.
\item [Initial spread limit.]
\index{initial spread limit}
This is $s_0$ in \cite{WilsonXCSR}.
It is the maximum initial spread when a new predicate is created through the covering operator.
\end{description}

\subsubsection{New in TSC}
These parameters are introduced here in TSC.

\begin{description}
\item [Maximum environment condition length.]
\label{sec:maximum-environment-condition-length}
\index{maximum environment condition length}
This is how many predicates we may have at the maximum in any individual classifier.
It should always be a positive integer.
\item [Maximum temporal mutation.]
\label{sec:maximum-temporal-mutation}
\index{maximum temporal mutation}
This is the most that the temporal element of the position may be randomly perturbed during the mutation process.
It should always be a positive integer.
\item [Maximum position mutation.]
\label{sec:maximum-position-mutation}
\index{maximum position mutation}
This is the most any dimensional element of a position may be randomly perturbed during the mutation process.
It should always be a positive integer.
\item [Valid operations.]
\index{valid operations}
This is a list of all the valid operations for the classifier, the $\omega$'s, a list of first-order lexical closures.
A first-order lexical closure is, roughly speaking, a function and its associated scope.
These $\omega$'s each must be capable of operating on any arbitrary list of data extracted from the data set, and these lists of data are extracted by following the raster paths through the data.
\item [Valid fields.]
\index{valid fields}
This is the list of valid fields for the classifier, the $\phi$'s, a list of first-order lexical closures.
These $\phi$'s must be capable of operating on a single time instance of the data.
\item [Visible time range.]
\index{visible time range}
This is the range in time that is visible to the classifiers.
None of the classifiers are allowed to look beyond this window.
This also is generally how much of a history should be generated before the classifier system is allowed to start.
This is a set interval.
\end{description}


\subsection{Trivially Modified Algorithms}
There are several algorithms from XCS and XCSR that are only slightly modified for our purposes from their original forms.

\begin{description}

\item[The \emph{Generate Match Set} Algorithm.]
This is the \emph{GENERATE MATCH SET} function in \cite{ButzWilson}.
The match set $M$ contains all of the classifiers in the population $P$ which match the current situation.  After filling the match set with all pre-existing matching classifiers, it repeatedly generates new covering classifiers until the minimum number of actions is satisfied.

\item[The \emph{Select Action} Algorithm.]
This is the same as in traditional XCS.
There are two methods for selecting an action used here: either randomly, or the best action.

\item[The \emph{Generate Action Set} Algorithm.]
This is the \emph{GENERATE ACTION SET} function in \cite{ButzWilson}.
It forms the action set $A$ out of the match set $M$, all of the classifiers that match the selected action.

\item[The \emph{Update Set} Algorithm.]
This is the \emph{UPDATE SET} function in \cite{ButzWilson}.
It updates the parameters for classifiers in the action set.

\item[The \emph{Update Fitness} Algorithm.]
\label{sec:update-fitness}
This is the \emph{UPDATE FITNESS} function in \cite{ButzWilson}.
The fitness of all of the classifiers in the action set are updated in a normalized manner.

\item[The \emph{Run GA} Algorithm.]
This is the \emph{RUN GA} function in \cite{ButzWilson}.
It runs a simple genetic algorithm, not on the full population $P$, but instead only on the action set $A$, in order to induce niching.

\item [The \emph{Select Offspring} Algorithm.]
This is the \emph{SELECT OFFSPRING} function in \cite{ButzWilson}.
It uses a roulette-wheel method of selection.

\item [The \emph{Insert into the Population} Algorithm.]
\index{insert into the population algorithm@\emph{insert into the population} algorithm}
This is the \emph{INSERT IN POPULATION} algorithm in \cite{ButzWilson}.
It is slightly more complex than just pushing the new classifier into the population list:
we need to check to see if it is already present in the population.
If it is, we must increment that classifier's numerosity instead.
For a new classifier $r$, find an $r' \in P$,
\index{population}
with $P$ being the entire population,
such that $r$ and $r'$ are identical.
If such an $r'$ exists, increment $r'_n$;
otherwise insert $r$ into $P$.

\item [The \emph{Delete from Population} Algorithm.]
This is the same as the \emph{DELETE FROM POPULATION} function in \cite{ButzWilson}.
It decides which members of the population are suitable for deletion, allowing for niching, and then removes low-fitness individuals.

\item [The \emph{Deletion Vote} Algorithm.]
\label{sec:deletion-vote}
\index{deletion vote algorithm@\emph{deletion vote} algorithm}
This is the same as the \emph{DELETION VOTE} algorithm in \cite{ButzWilson}.
The deletion vote for a classifier $r$ is dependent upon its action set size estimate.
\index{action set size estimate}
Let $F_{average}$ be the average fitness in the entire population.
We want classifiers with sufficient experience and a significantly lower than average fitness than the rest of the population to be deleted before others.
Expressed in terms of the TSC parameters as outlined in \S\ref{sec:parameters}:
\begin{equation}
r_{exp} > \theta_{del} \bigwedge \frac{r_F}{r_n} < \delta F_{average}.
\end{equation}
\index{experience}
\index{deletion threshold}
\index{GA fitness}
\index{numerosity}
This then returns
\begin{equation}
\frac{r_{as} r_n F_{average}}
   {\left( \frac{r_F}
               {r_n} \right)}
= 
\frac{r_{as} r_n^2 F_{average}}
   {r_F}
\end{equation}
\index{action set size estimate}
as the deletion vote for this classifier $r$;
otherwise it returns
$r_{as} r_n$
as the deletion vote for this classifier $r$.

\item [The \emph{Do Action Set Subsumption} Algorithm.]
This is the \emph{DO ACTION SET SUBSUMPTION} function in \cite{ButzWilson}.
The function chooses the subsumer from the most general classifiers capable of subsumption and then subsumes all possible classifiers in to the subsumer.

\item [The \emph{Could Subsume?} Predicate.]
\label{sec:could-subsume?}
We say that a specific classifier $r$ is capable of subsuming others if it has both sufficient accuracy and sufficient experience.
That is, if the experience of the classifier is greater than the
minimal subsumption experience threshold\index{minimal subsumption experience},
and the prediction error of the classifier is less than the equal error threshold.
In symbols:
\begin{equation}
r_{exp} > \theta_{sub} \bigwedge r_{\epsilon} < \epsilon_0.
\end{equation}

\item [The \emph{Subsume?} Predicate.]
This is called \emph{DOES SUBSUME} in \cite{ButzWilson}.
A classifier $r^1$ subsumes another classifier $r^2$ if the following conditions are all met:
\begin{algorithmic}[1]
\STATE Their actions are identical: $r^1_a = r^2_a$.
\STATE The classifier $r^1$ is capable of subsumption, as decided by the \emph{could subsume?} predicate described in \S\ref{sec:could-subsume?}.
\STATE The classifier $r^1$ is more general than the classifier $r^2$, as decided by the \emph{more general?} predicate described in \S\ref{sec:more-general?}.
\end{algorithmic}

\end{description}

\subsection{The \emph{Match?} Predicate}
\label{sec:match?}
\index{match? predicate@\emph{match?} predicate}
This is based upon the algorithm called \emph{DOES MATCH} in \cite{ButzWilson}, but it has been generalized in order to suit our needs here.
Assume a classifier $r$ and a situation $\sigma$.
In traditional learning classifiers, $\sigma \in \{false, true\}$ which is usually represented $\{0,1\}$,
and therefore it is only necessary to see if every element in the condition part of the classifier $r$,  that is $r_c$, is either equal to each other or a covering symbol in $r$:
\begin{equation}
   \left( r_{c_i} = \sigma_i \bigvee r_{c_i} = \# \right)
      \forall i \in \mathbb{Z}_{|r_c| = |\sigma|}.
\end{equation}
For us, it is slightly more involved due to the more complex nature of the conditions used in the construction of the classifiers.

\begin{description}

\item [The \emph{match?} predicate for ternary values.]
For ternary values as used in traditional learning classifiers, a ternary predicate $t$ matches a situation element $x$ when either
$t = x$ or $t = \#$, the covering symbol.
Similarly, a ternary predicate $t$ matches a second ternary predicate $u$ when $t$ matches all of the situations matched by $u$;
that is, when $t = u \bigvee t = \#$.

\item [The \emph{match?} predicate for ranges.]
For ranges as used in Wilson's XCSR \cite{WilsonXCSR}, a range predicate $r$ matches a situation $x$ when that situation $x$ lies within the lower and upper bounds specified by the range predicate, $l \le x \le u$.

\item [The \emph{match?} predicate for a time-series.]
If we take the data along the straight line segment from the initial point $A$ to the final point $B$, forming a vector $d$, we can then form $d'$ by applying $\phi$ to each element in $d$:
\begin{equation}
d'_i = \phi \left(d_i\right) \forall d_i \in d.
\end{equation}
The predicate is said to match the data if and only if
\begin{equation}
l \le \omega \left( d' \right) \le u.
\end{equation}
When all of the predicates of the rule match, then the rule matches; the rule then recommends a particular classification or action.

Two situations $\sigma_1$ and $\sigma_2$ match if every one of their elements match element-wise:
\begin{equation}
   \mathrm{match?}\left( \sigma_{1_i}, \sigma_{2_i} \right) = true \;
      \forall i \in \mathbb{Z}_{|\sigma_1| = |\sigma_2|}.
\end{equation}

\item [The \emph{match?} predicate for classifiers and situations.]
A classifier $r$ matches a situation $\sigma$
if $r^1$ and $r^2$ match,
as decided by the \emph{match?} predicate described in \S\ref{sec:match?},
and at least one of the elements of the classifier is more general in $r^1$ than in $r^2$.

\item [The \emph{match?} predicate for classifiers.]
A classifier $r^1$ matches another classifier $r^2$
if the environment condition of $r^1$ matches the environment condition of the classifier $r^2$.

\end{description}

\subsection{The \emph{Generate Covering Classifier} Algorithm}
This is derived from the \emph{GENERATE COVERING CLASSIFIER} function in \cite{ButzWilson}.
It creates a classifier which matches the current situation.
This is handled somewhat differently in TSC than in XCS or in XCSR, and the method operates as described in Algorithm~\ref{alg:generating-covering-classifiers}.
\begin{algorithm}
\caption{Generating covering classifiers.}
\begin{algorithmic}[1]
\label{alg:generating-covering-classifiers}
\INPUT a TSC instance.
\LET $l$ be randomly chosen, $1 \le l \le $ the maximum environment condition length.
\LETARROW{$c$, the condition} $nil = \{\}$, an empty list.
\LETARROW{$a$, the action} a random element from the set of all possible actions that are not in the match set.
\FOR{$l$ times}
  \PUSH{a covering predicate}{$c$}
\ENDFOR
\RETURN a new classifier instance with environment condition $c$, action $a$, time stamp set to the current number of situations, and the rest of the slots set to their defaults. 
\end{algorithmic}
\end{algorithm}

%\subsection{The \emph{Crossover} Algorithm}
%This serves the same purpose as \emph{APPLY CROSSOVER} in \cite{ButzWilson}.
%The method of crossover used is described in \S\ref{sec:crossover} here.

%\subsection{The \emph{Mutate} Algorithm}
%\index{mutate algorithm@\emph{mutate} algorithm}
%This serves the same purpose as the \emph{APPLY MUTATION} algorithm in \cite{ButzWilson}.
%The method of mutation used is described in \S\ref{sec:mutation} here.

\subsection{The \emph{More General?} Predicate}
\label{sec:more-general?}
This is derived from the \emph{IS MORE GENERAL} function in \cite{ButzWilson}.

\begin{description}
\item [The \emph{more general?} predicate for a TSC predicate.]
This returns true only if the predicate $p$ matches predicate $q$ and if it is more general than it as well.
Predicate $p$ is more general than predicate $q$ if and only if:
\[p \textrm{ matches } q \land\]
\[ \left( l_p < l_q \lor u_q < u_p \lor
      \left( path_q \textrm{ lies completely along } path_p \land path_p \neq path_q\right)
   \right)
.\]

\item [The \emph{more general?} predicate for classifiers.]
This is based upon the algorithm called \emph{IS MORE GENERAL} in \cite{ButzWilson}, but it has been generalized in order to suit our needs here.
In traditional learning classifiers, it is only necessary to count the occurrences of the covering symbol, $\#$, in order to determine which of two classifiers is more general: the one with the greater number of occurrences of it.
For us it is slightly more involved due to the more complex nature of the conditions used in the construction of the classifiers.
A classifier $r^1$ is more general than another classifier $r^2$ if $r^1$ and $r^2$ match,
as decided by the \emph{match?} predicate described in \S\ref{sec:match?},
and at least one of the elements of the classifier is more general in $r^1$ than in $r^2$.

\end{description}

